\documentclass{manual}
\usepackage{hyperref}

\title{ANUGA What's New}

\author{Nariman Habili}

% Please at least include a long-lived email address;
% the rest is at your discretion.
\authoraddress{Geoscience Australia \\
  Email: \email{anuga@ga.gov.au}
}

%Draft date
\date{\today}   % update before release!
                % Use an explicit date so that reformatting
                % doesn't cause a new date to be used.  Setting
                % the date to \today can be used during draft
                % stages to make it easier to handle versions.

		
%
% This file is a dummy used if update_anuga_user_manual.py hasn't 
% generated one.
%	
	

% release version; this is used to define the
% \version macro
\release{1.3.1}
 % Get version info - this file may be modified by
                % update_anuga_user_manual.py - if not a dummy 
		% will be used.
				

\makeindex          % tell \index to actually write the .idx file
%\makemodindex          % If this contains a lot of module sections.



\begin{document}
\maketitle



% This makes the contents more accessible from the front page of the HTML.
\ifhtml
\chapter*{Front Matter\label{front}}
\fi




\chapter{ANUGA release information}

This document outlines major bug fixes and new functionality added to ANUGA between releases.
These lists are not comprehensive and we refer to the TRAC system for a complete audit trail of changes made to ANUGA.
\begin{itemize} 
    \item Timeline of all changes to the code base: https://datamining.anu.edu.au/anuga/timeline
    \item Changesets between two releases: To see all changes between subversion versions 4669 and 4733, for example, visit:
    \begin{itemize} 
      \item \url{https://datamining.anu.edu.au/anuga/changeset/4669} through to
      \item \url{https://datamining.anu.edu.au/anuga/changeset/4733}
    \end{itemize}   
\end{itemize}     

The release names take the form
\begin{verbatim}
  AA-BB.CC.DD.EXT
\end{verbatim}
where
\begin{itemize}
  \item \code{AA} is the name of the anuga component, e.g. 
  \code{anuga}, \code{anuga\_viewer}, \code{anuga\_installation\_guide} or \code{anuga\_user\_manual}
  \item \code{BB} is the major revision number. The major revision number is unlikely to change very often unless the code has undergone a major change.
  \item \code{CC} is the minor revision number. This number will increase when the code undergoes major bug fixes, changes to the interface, optimisation and the addition of minor features.
  \item \code{DD} is the minor revision number. This number will increase when the code undergoes minor bug fixes.
  \item \code{EXT} is the file name extension \code{tgz}\footnote{Internet explorer has the habit of renaming the .tgz files to .gz - the remedy is to rename them back or use another browser such as Firefox.} used for source code or \code{pdf} used for documentation.
\end{itemize}




\chapter{Release notes}

This chapter list main developments between releases. Small bug fixes, 
refactoring, documentation updates, style fixes are generally not reported 
here. Refer to the Subversion 
log (https://datamining.anu.edu.au/anuga/timeline) and closed tickets
https://datamining.anu.edu.au/anuga/report/10) for all changes in ANUGA.

\section{Bleeding Edge}

Work on parallel ANUGA, Kinematic Viscosity, 1-D pipeflows and wind/pressure
forcing terms. There is also activity in the area of viewers and installers.

\section{Release Name: anuga-1.2.0, Date: July 2010}
Breaks code compatibility with older ANUGA versions. Please see the ANUGA wiki for migration instructions.
Simplified API, with more logical module locations and names. 
Support for internal boundaries and holes in meshes.
Support for user-specified breaklines.
Speed optimisations - fitting is around 25 percent faster.
Various bug fixes - see trac. Note the new versioning system.


\section{Release Name: anuga-1.1beta\_7607, Date: 12 January 2010}
Discarded obsolete variable 'z' in sww files. This commits ANUGA to the 
new viewer: \url{http://www.ausposdevelop.com.au/trac/anuga_viewer}.
Small optimisations and cleaning
Intoduced one-click Windows installer
Updates on the parallel code.

\section{Release Name: anuga-1.1beta\_7472, Date: 3 September 2009}

Ability to store any quantity in the sww file either as static or 
time dependent. This allows for example storage of variable bed elevation 
and/or friction. See manual for details on set\_quantities\_to\_be\_stored.


\section{Release Name: anuga-1.1beta\_7315, Date: 20 July 2009}

ANUGA updated to work with Python 2.6
Reference to new viewer in manual.



\section{Release Name: anuga-1.1beta\_7302, Date: 6 July 2009}
This is the first release of ANUGA based on the package Python-numpy.
The reason for this upgrade is that the old package Numeric is 
no longer supported and doesn't work for some systems.
See the installation guide for details regarding the new dependencies.


\section{Release Name: anuga-1.0beta\_7163, Date: 9 June 2009}

THIS IS THE LAST OFFICIAL RELEASE OF ANUGA BASED ON THE PYTHON NUMERIC PACKAGE.

\begin{itemize}
  \item Several optimisations. ANUGA now runs at least 10-15\% faster overall. See changesets 7143, 7136, 7105, 7034, 6840, 6737, 6703.
  \item Culverts based on the Boyd method have been refactored and test suite added thanks to Rudy van Drie and Petar Milevski.
  \item Added a special purpose boundary (AWI boundary) provide by Nils Goseberg.
  \item Cleanup Cairns demo and introduced example of new method add\_quantity.
\end{itemize}



\section{Release Name: anuga-1.0beta\_6838, Date: 19 April 2009}
\begin{itemize}
  \item Implemented availability of average energy as well as flow 
  through cross-section at run-time (ticket:295)
  \item Stored permutation from urs2sts in STS file (ticket:298) 
  \item Equipped File\_boundary and Field\_boundary with default boundary
  conditions to allow modelling beyond time interval (ticket:293) 
  \item Allowed fitting to reuse ANUGA mesh saving memory and time.
  \item Started new module anuga.interface containing common functions. In particular the new function create\_domain\_from\_regions (ticket:308 and changeset:6190) which can replace the old call to create\_mesh\_from\_regions followed by Domain. This not only simplifies domain creation, it also makes caching of this process water proof. This function has not yet been documented in the user manual.
  \item Made evolve loop about 5\% faster due to optimisations in 
  changeset:6703 and changeset:6737.
  \item Fixed tests that only failed on Windows. 
\end{itemize} 


\section{Release Name: anuga-1.0beta\_5638, Date: 11 August 2008}

\begin{itemize}
  \item Better diagnostics for timestepping. 
  \item Implemented tracking of IP for data files bundled with ANUGA to ensure that all are legally OK to distribute.
  \item Improved logging of model runs (screen_catcher and copy_code_files).
  \item Refactored graphing of timeseries into extraction and plotting (see sww2timeseries and sww2cvs\_gauges).
  \item Improved performance and memory management in generate mesh and  least squares fitting.
  \item Simplified Quantity data structure.
  \item Added more validation examples. 
  \item Function for automatically determining optimal smoothing parameter through cross validation (\code{get\_flow\_through\_cross\_section}, p 57 in the user manual)
  \item Implemented better second order approximation through the option to use edge limiters along with second order Runge-Kutta timestepping. 
        This provides better accuracy in some cases (e.g. waves in deep water over long distances)
  \item Made tight\_slope\_limiters the default.
  \item Retired obsolete parameter beta_h
  \item Added the Okada tsunami source model as an optional initial condition in ANUGA
  \item Upgraded ANUGA to work with Python 2.5
  \item New fileboundary (using NetCDF format with extension .sts) coupling for timeseries on a list of points.
  \item Added new forcing terms for flood modelling capability: Rainfall, Inflow, Culverts, ... (See p 51-53)
\end{itemize}   
  
  
\section{Release Name: anuga-1.0beta\_4824, Date: 15 Nov 2007}

\begin{itemize} 
  \item Removal of obsolete Python code where (faster) C code exists. 
  \item Several updates in the documentation in response to postings. 
  \item Improved installation and compilation procedure. 
  \item Addressed excessive memory use in fitting (currently optional as it appears somewhat slower) 
\end{itemize} 


\section{Release Name: anuga-1.0beta_4733, Date 12 Sep 2007}

\begin{itemize} 
  \item A number of optimisations making the evolution part of ANUGA about 40\% faster. See \url{https://datamining.anu.edu.au/anuga/ticket/135} for details. 
  The optimisations are 
  \begin{itemize} 
     \item Dry cell exclusion from flux calculations and linear reconstruction of triangles. 
     This optimisation will be most effective in domains with large dry areas.  
     \item Separation of functions into gateways and computational routines 
     \item Utilisation of static work arrays 
     \item A large number of minor optimisations 
  \end{itemize} 
  
  \item Obsolete code was cleared out. 
\end{itemize}   


\section{Release Name: anuga-1.0beta_4669, Date 17 Aug 2007}
\begin{itemize} 
  \item Improved speed in set\_quantity
  \item deprecated xya file format
  \item general maintenance
\end{itemize}    
  
\section{Release Name: anuga-1.0beta_4530, 4 June 2007}

\section{Release Name: anuga-1.0beta_4492, 25 May 2007}

\section{Release Name: anuga_1.0beta_4106, 20 Dec 2006}

\begin{itemize} 
  \item First public release of ANUGA - Hydrodynamic Modelling.
\end{itemize} 
Version 1.0beta is the first version for general use. It is considered as a beta release as we expect feedback and suggestions for improvement by the community.

The minor release number (4106) is the revision number from Subversion and uniquely defines the exact version of ANUGA.


\end{document}
