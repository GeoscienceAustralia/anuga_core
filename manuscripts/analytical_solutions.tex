\documentclass[11pt]{article}
\usepackage{amssymb}		% to get all AMS symbols
\usepackage{amsmath}		% to get all AMS symbols
\usepackage{cases}
\usepackage{siunitx}
\newcommand{\me}{\mathrm{e}}
\begin{document}

\title{Testing Anuga-Sed against analytical and experimental solutions}
\maketitle

We wish to test Anuga-Sed against analytical solutions for its equations as well as agaisnt experimental datasets for sediment transport, flow through vegetation, and sediment transport through vegetation. This will confirm that the model can simulate these processes with a certain degree of accuracy before using the model for more complex simulations and numerical experiments.

\section{Sediment transport}

\subsection{Analytical solution to transport equations}

The governing equation for sediment transport in Anuga-Sed is the mass balance of sediment in the water column and bed layer:

\begin{equation}
\frac{\partial C h}{\partial t} = \dot{E} - \dot{D} - \left(\frac{\partial q_{s_x}}{\partial x} + \frac{\partial q_{s_y}}{\partial y} \right)
\end{equation}

\noindent where $h$ is the flow depth, $\dot{E}$ is entrainment flux, $\dot{D}$ is deposition flux, $C$ is the sediment concentration, and $q_{s_x}$ and $q_{s_y}$ are specific volumetric sediment fluxes in the $x$ and $y$ directions \citep[e.g.,][]{davy2009fluvial}. Sediment flux is given by $q_{s_{x,y}} = \beta C q_{x,y}$, where $q_{x,y}$ is specific discharge in the $x$ or $y$ direction and $\beta$ ($\le 1$) describes the effective speed of sediment relative to that of the water.

Assuming that transport of water on sediment only occurs in $x$, the equation can be simplified to:

\begin{equation}
\frac{\partial C h}{\partial t} = \dot{E} - \dot{D} - \frac{\partial q_{s_x}}{\partial x}
\end{equation}

This assumption is reasonable for simulations of flow over ramps that slope only in the $x$ direction. By assuming that the volume of sediment in the water column had reached a steady state, we further simplified the mass balance to write:

\begin{equation}
\frac{dq_{sx}}{dx} = \dot{E} - \dot{D}
\end{equation}

Given $q_{sx} = Cq_x$, we can write

\begin{equation}
\frac{dC}{dx} = \frac{\dot{E}}{q_x} - \frac{\dot{D}}{q_x}
\end{equation}

The deposition rate is $\dot{D} = d^*v_sC$, where $d^*$ is a parameter that describes the distribution of sediment in the water column and $v_s$ is the settling velocity of the sediment. Both of these can be assumed to be constants (or near constants).

Integrating with respect to distance downstream $x$, we find an expression for the volumetric sediment concentration with distance downstream from the sediment source.

\begin{equation}
C(x) = \left(C_o - \frac{\dot{E}}{d^*v_s}\right)\me^{-\frac{d^* v_s x}{q}} + \frac{\dot{E}}{d^* v_s}
\end{equation}

\noindent where $C_o$ is the sediment concentration at the inlet.

The rate of change of bed elevation $\eta$ is given by

\begin{equation}
\frac{d\eta}{dt} = \frac{\dot{D} - \dot{E}}{1 - \phi}
\end{equation}

\noindent where $\phi$ is the bed porosity. Combining both equations we find:

\begin{equation}
\eta(x,t) = \frac{t}{1-\phi} \left(d^* v_s C_o - \dot{E}\right) \me^{-\frac{d^* v_s x}{q}} + \eta_o(x)
\end{equation}

\noindent where $\eta_o(x)$ is the initial profile of the bed.


\begin{table}[]
\centering
\caption{Model constants}
\label{table:model_params}
\begin{tabular}{ll}
$\phi$ & 0.3 \\
g & 9.81 \\
$\kappa$ & 0.408 \\
$\rho_w$ & 1000 \\
${\tau_c}^*$ & 0.06 \\
${\tau_c}$ & 0.126126 \\
$v_s$ & 0.0138099680617
\end{tabular}
\end{table}

Table \ref{table:model_params} summarizes the constants used in these simulations.



\begin{table}[]
\centering
\caption{Simulation "plane3"}
\label{table:plane3_params}
\begin{tabular}{|l|l|l|}
\hline
Flow algorithm       & DE0             &                       \\ \hline
Length               & 15              &                       \\ \hline
Width                & 10               &                       \\ \hline
dx, dy               & 0.5             &                       \\ \hline
Initial topography   & $z = 10 - x/50$ &                       \\ \hline
Initial stage        & topography + 0.4      &                       \\ \hline
Friction             & 0.0             &                       \\ \hline
Inflow boundary      & {[}10.42,8,0{]}    & Dirichlet boundary    \\ \hline
Side boundaries      &                 & Transmissive boundary   \\ \hline
Outflow boundary     &                 & Transmissive boundary \\ \hline
Inflow concentration & 0.005           &                       \\ \hline
Yieldstep            & 1               &                       \\ \hline
Finaltime            & 60              &                       \\ \hline
\end{tabular}
\end{table}


We ran a simulation "plane3" with the parameters summarized in table \ref{table:plane3_params}. Soon after the start of the simulation, the volume of sediment in the water column had reached steady state (defined as maximum and mean differences in sediment volume between two consecutive timesteps of less than 0.0001).

It is necessary to use transmissive boundaries for all edges (except the inflow) to avoid reflecting flow and sediment on the edges. Reflective boundaries decrease the slope of the bed by increasing momentum near the inlet. For all calculations, only values of the quantities in the range $4<y<6$ were included. Quantities were interpolated using the SciPy Savitzky?Golay filter with a window size of 7 and 3rd order polynomial to an array with 45 points evenly spaced between $x = 0.1$ and $x = 14$.

The value of $d^*$ calculated by the model was exported at every timestep. The profile of $d^*$ strongly influences the profile of the steady state concentration in the analytical solution.

The initial bed profile was the topography at timestep 0. The concentration at $x=0$ was calculated for every timestep as the maximum concentration at that timestep.







\end{document}
