\documentclass[11pt]{article}
\usepackage{cases}
\begin{document}

\title{Implementation of Infiltration operator in anugaSed}
\date{}
\maketitle

An infiltration operator was implemented in anugaSed in order to simulate flows in the Colorado River Delta, which is a strongly losing stream.

We implemented the Green-Ampt equation for infiltration (Green and Ampt, 1911) because it is computationally more efficient than the Richards equation. The Green-Ampt equation makes some simplifying assumptions:

\begin{itemize}
\item The wetting front moves down into dry soil
\item Matric suction pulls water into dry soil
\item The soil above the wetting front is saturated
\end{itemize}

The simplest form of the Green-Ampt equation for infiltration rate, $f$ can be written as:

\begin{equation}
f = -K_s \frac{dh}{dz}
\end{equation}

\noindent where $dh/dz$ is the hydraulic gradient and $K_s$ is the saturated hydraulic conductivity. This can be rewritten as:

\begin{equation}
f = -K_s \frac{h_f - h_o}{Z_f}
\end{equation}

\noindent where $h_f$ is the hydraulic head at the wetting front (sum of matric forces at the wetting front and the weight of the water above), $h_o$ is the hydraulic head at the surface (zero, unless there is water ponded on the surface), and $Z_f$ is the depth of the wetting front. The depth of the wetting front can be related to the cumulative amount of infiltrated water $F$ by:

\begin{equation}
F = Z_f (\theta_s - \theta_i)
\end{equation}

\noindent where $\theta_s$ is the saturated moisture content and $\theta_i$ is the initial moisture content before infiltration began. Combining these equations, we can write a relationship for the infiltration rate (after water has started ponding at the surface):

\begin{equation}
f(t) = - K_s \frac{h_f - h_o - Z_f}{Z_f}
\end{equation}

A simpler approach is to recognize that, during an ephemeral flow event, the stream channel surface will quickly become saturated and a positive pressure head, equation to the stream stage, will be exerted at the channel boundary (Freyberg et al., 1980). The velocity of the wetting front can be extressed as:

\begin{equation}
v_f = \frac{f(t)}{\theta_s - \theta_i} = -K_s \frac{h_f - h_o - Z_f}{Z_f (\theta_s - \theta_i)}
\end{equation}

The time it takes the wetting front to advance a depth $Z_f$ is then:

\begin{equation}
t = \frac{(\theta_s - \theta_i)}{K_s} \left[ Z_f + (h_f - h_o) \; ln \left[1 + \frac{Z_f}{h_o - h_f}\right]\right]
\end{equation}

The rate of advance of the wetting front eventually decreases to a constant value of $K_s / (\theta_s - \theta_i)$.

From Processes Controlling Recharge Beneath Ephemeral Streams in Southern Arizona - Kyle Blasch and Ty Ferre, in Groundwater Recharge in a Desert Environment


Green-Ampt parameters as determined by Rawls/Brakensiek (1993) (Hydrology Handbook) for sand

Porosity: 0.437 (0.374 - 0.5)
Wetting front soil suction head (cm): 4.95 (0.97 - 25.36)
Saturated hydraulic conductivity (cm/hr): 23.56

From Dingman: Physical Hydrology

After ponding, infiltration rate is given by Darcy's law:

\begin{equation}
f(t) = K_s - K_s \frac{\psi_f + h_o}{Z_f}
\end{equation}

\noindent where $\psi_f$ is the effective tension at the wetting front and $h_o$ is the depth of ponding.

\subsection{Implementation}

A quantity is created called "wetting front depth" that stores the depth to the wetting front (in meters) for every cell in the landscape. By default, the initial value of this quantity is 0.5 cm (0.0005 m). This is done to avoid calculating infinite infiltration rates.

At every timestep in the simulation, the operator calculates the infiltration rate $f$ for all cells with a minimum flow depth of 1 cm using the equation:

\begin{equation}
f = -K_s \frac{\psi_f - h_o - Z_f}{Z_f}
\end{equation}

\noindent where $\psi_f$ is the effective tension at the wetting from, $h_o$ is the depth of flow, and $Z_f$ is the value of wetting front depth. The infiltration rate is multiplied by the timestep to obtain the streamflow loss (in meters). This value is subtracted from the flow depth.

The value of the quantity wetting front depth is updated by calculting the velocity of the wetting front:

\begin{equation}
v_f = \frac{f}{(\theta_s - \theta_i)}
\end{equation}

\noindent where $\theta_s$ and $\theta_i$ are the saturated moisture content and the initial moisture content, respectively. The value of $v_f$ is multiplied by the timestep to obtain the change in depth of the wetting front, and the quantity wetting front depth is updated.



\end{document}
