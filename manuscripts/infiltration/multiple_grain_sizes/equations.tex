\documentclass[11pt]{article}
\begin{document}

\title{Equations for calculating sediment transport with multiple grain sizes}
\date{}
\maketitle

For using multiple grain sizes, the sediment transport operator must receive a list of grain sizes (in mm), and a quantity called "grain size fractions" with a numpy ndarray of the fraction of each grain size in the bed material throughout the landscape. The model will assume that this is the composition of the sediment at all depths below the initial topography (the composition of all pre-existing erodible material).

Also list of critical shear stresses (how to compute?).

The operator currently assumes that all sediment eroded from the surface has the composition of the pre-existing material, even if sediment with a different composition has deposited on the landscape. This is done for efficiency.

The transport of bedload and suspended sediment is governed by different equations. At every cell, each grain size is determined to be either bedload or suspended load by calculating their Rouse number:

\begin{equation}
P = \frac{w_s}{\kappa u_*}
\end{equation}

Fractions with $P > 7.5$ are immobile, those with $2.0 < P < 7.5$ are treated as suspended load, and those with $P < 2.0$ are considered bedload.


\subsection{Bedload}

We use the Wilcock and Crowe (2003) formulation for sediment mixtures to calculate the sediment discharge of any bedload fractions. The dimensionless transport parameter $W_i^*$ is given by:

\begin{equation}
W_i^* = \frac{R g q_{bi}}{F_i {u_*}^3}
\end{equation}

\noindent where $R$ is the submerged specific gravity of sediment, $q_{bi}$ is the bedload discharge for grain size $i$, $F_i$ is the fraction of grain size $i$ on the bed, and $u_*$ is the shear velocity. The dimensionless transport parameter takes different values depending on the transport stage $\phi$:

\begin{equation}
\phi = \frac{\tau_b}{{\tau_c}_i}
\end{equation}

\noindent where $\tau_b$ is the shear stress applied by the flow on the bed and ${\tau_c}_i$ is the critical shear stress for grain size $i$. If $\phi < 1.35$, $W_i^*$ is equal to:

\begin{equation}
W_i^* = 0.002 \phi^7.5
\end{equation}

If $\phi \ge 1.35$, $W_i^*$ is:

\begin{equation}
W_i^* = 14 \left( 1 - \frac{0.894}{\phi^0.5} \right)^4.5
\end{equation}

\subsection{Suspended load}

We use the formulation of Wright and Parker (2004) for sediment mixtures as presented in Gary Parker's e-book (chapter 10). The discharge of suspended sediment is given by:

\begin{equation}
q_{si} = \frac{u_* E_i H}{\kappa} I
\end{equation}

\noindent where $E_i$ is the entrainment rate for grain size $i$ and $H$ is the flow depth. The term $I$ is given by:

\begin{equation}
I = \int_{\delta_b}^{1} {\left[ \frac{(1 - \delta) / \delta}{(1 - \delta_b) / \delta_b} \right] } ^{w_s/\kappa u_*}   \ln \left( \frac{30 H}{k_c} \delta \right) d\delta
\end{equation}

\noindent where $\delta = z / H$ and $\delta_b = b / H$ and $b$ is the roughness height. Following McLean (1992), we use $b = \alpha_o h_b$, where $h_b$ is the bedload layer thickness and $\alpha_o$ is a constant $\alpha_o = 0.056$. The bedload layer thickness is given by equation 16 in McLean (1992).

The first term inside the integral is the Rouse equation, $\frac{\bar{C}}{C_a}$. The term $k_c$ is a composite roughness height. We assume that bedforms are absent, so $k_c = n_k D_{90}$, where $n_k$ is a constant between 1 and 2.

\end{document}
