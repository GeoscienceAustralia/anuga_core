\documentclass[11pt]{article}
\usepackage{amssymb}		% to get all AMS symbols
\usepackage{amsmath}		% to get all AMS symbols
\usepackage{cases}
\usepackage{siunitx}
\begin{document}

\title{Governing equations for ANUGA sediment transport}
\maketitle

\section{Hydrodynamics}

ANUGA is a fluid flow model originally developed to simulate sudden releases of water such as tsunamis and dam breaks into urban environments.

This hydrodynamic model primarily differs from the kinematic wave model in three ways: (a) it allows for rapidly varying flow depth and velocity ,instead of assuming steady, uniform flow, (b) it routes runoff across hillslopes using the same equations of flow as in the channels, instead of using empirical relationships and lumping the hillslope inflow parameters, and (c) it computes the two dimensional distribution of streamflow across the landscape, instead of treating the channel networks as a series of one dimensional channel segments.

ANUGA solves the conserved form of the 2-D, depth-averaged Shallow Water Wave equations, which describe the flow of a thin layer of fluid over a surface:

\begin{equation}
\frac{\partial{\mathbf{U}}}{\partial{t}}+\frac{\partial{\mathbf{E}}}{\partial{x}}+\frac{\partial{\mathbf{G}}}{\partial{y}} = \mathbf{S}
\end{equation}

\noindent where $\mathbf{U} = [h\;uh\;vh]^T$ is the vector of conserved quantities depth $h$, $x$-momentum $uh$, and $y$-momentum $uh$. $\mathbf{E}$ and $\mathbf{G}$ are the fluxes in the $x$ and $y$ directions and $\mathbf{S}$ is the source term:

\begin{equation}
 \mathbf{E} = \begin{bmatrix}
       uh\\[0.5em]
       u^2h+gh^2/2\\[0.5em]
       uvh\\
     \end{bmatrix}
     \qquad
      \mathbf{G} = \begin{bmatrix}
       vh\\[0.5em]
       vuh\\[0.5em]
       v^2h+gh^2/2\\
     \end{bmatrix}
 \qquad
      \mathbf{S} = \begin{bmatrix}
       0\\[0.5em]
       -gh(z_x + S_{fx}\\[0.5em]
       -gh(z_y + S_{fy})\\
     \end{bmatrix}
\end{equation}
\ \\
\noindent where $z$ is the bed elevation above a datum and $S_f$ is the bed friction slope:

\begin{equation}
S_{fx} = \frac{un\sqrt{u^2 + v^2}}{h^{4/3}}
 \qquad
 S_{fy} = \frac{vn\sqrt{u^2 + v^2}}{h^{4/3}}
\end{equation}

ANUGA solves these equations using a Godunov-type finite-volume scheme with a first-order approximate Riemann solver \citep{toro1992riemann} and explicit Euler timestepping method with variable timesteps on an irregular triangular grid. Because it allows for discontinuities of the conserved quantities across the edges of the cells, it can resolve arbitrarily steep wave fronts as well as hydraulic shocks. Capturing flow discontinuities is necessary when simulating the dynamics of flash floods in ephemeral channels since these typically have a steep leading edge or 'bore' (Leopold and Miller, 1956) that must be modeled correctly in order to capture the peak flow stage and velocity (Benzvi et al., 1991; Pilgrim, 1976) and maintain mass balance (Garcia-Navarro et al., 1999; Mudd 2006).

Anuga-Sed requires the use of one of Anuga's discontinuous elevation flow algorithms. The preferred algorithm is 'DE0', which uses first-order timestepping. This algorithm is the fastest of the DE algorithms implemented in the code. These algorithms introduce some error in the position of the surface because they allow for some movement of the bed surface during the flow calculation to preserve numerical stability. This topographic change does not represent "real" change and introduces some noise in the model result but it is generally negligible. We recomment running the model and recording elevation at every timestep without the sediment transport operator to assert that this error is small for your particular parameter space.

\section{Sediment transport}

Large floods can alter the shape of an easily erodible surface rapidly enough to drastically change the flow field during the event. Simulating these events therefore demands the dynamic evolution of the model topography in response to the computed flow and sediment transport.

The strong spatial variability in shear stress caused by complex flow patterns and variability in drag resulting from spatially variable vegetation cover requires the calculation of an explicit sediment mass balance within the water column in order to handle the local disequilibria between grain entrainment and settling rates \citep{davy2009fluvial}. The mass balance for sediment in the water column and moving bed layer is written as

\begin{equation}
\frac{\partial C h}{\partial t} = \dot{E} - \dot{D} - \left(\frac{\partial q_{s_x}}{\partial x} + \frac{\partial q_{s_y}}{\partial y} \right)
\end{equation}

\noindent where $\dot{E}$ is entrainment flux, $\dot{D}$ is deposition flux, $C$ is sediment concentration, and $q_{s_x}$ and $q_{s_y}$ are specific volumetric sediment fluxes in the $x$ and $y$ directions \citep[e.g.,][]{davy2009fluvial}. Sediment flux is given by $q_{s_{x,y}} = \beta C q_{x,y}$, where $q_{x,y}$ is specific discharge in the $x$ or $y$ direction and $\beta$ ($\le 1$) describes the effective speed of sediment relative to that of the water.

The elevation of the surface $z$ is given by

\begin{equation}
\frac{\partial z}{\partial t} = \frac{\dot{D} - \dot{E}}{1 - \phi}
\end{equation}

\noindent where $\phi$ is the porosity of the bed material.

\subsection{Entrainment}

%The entrainment flux $\dot{E}$ is (e.g. Hanson, 1990):
%
%\begin{equation}
%\dot{E} = K_e \: (\tau^* - {\tau_c}^*)
%\end{equation}
%
%\noindent where $K_e$ is an erosion coefficient and ${\tau_c}^*$ is the dimensionless critical shear stress. The erosion coefficient $K_e$ is given by:
% 
%\begin{equation}
%K_e = {K_e}^* \: D_{50} \: \sqrt{R \: g \: D_{50}}
%\end{equation}
%
%\noindent where ${K_e}^*$ is a dimensionless erosion coefficient. $\tau^*$ is the dimensionless shear stress that the flow applies on the bed:
%
%\begin{equation}
%\tau^* = \frac{{u_*}^2}{R \: g \: D_{50}}
%\end{equation}
%
%\noindent where $u_*$ is the shear velocity, $D_{50}$ is the median grain diameter, $R = (\rho_s - \rho_w) / \rho_w$ is the submerged specific gravity of sediment, $\rho_w$ and $\rho_s$ are the densities of water and sediment, and $g$ is the gravitational acceleration.
%
%%The shear velocity $u_*$ is given by:
%%
%%\begin{equation}
%%u_* = \frac{\langle u \rangle \kappa}{\log{(h / z_o)} - 1}
%%\end{equation}
%%
%%\noindent where $\langle u \rangle$ is the depth-averaged flow velocity, $\kappa$ is Von Karman's constant ($\kappa = 0.4$) and $z_o = D_{50}/30$. The expression for $u_*$ is derived from the Law of the Wall equation for depth-averaged flow velocity:
%%
%%\begin{equation}
%%\langle u \rangle = \frac{u_*}{\kappa} \bigg[ \ln{\bigg(\frac{h}{z_o}\bigg)} - 1 \bigg]
%%\end{equation}
%
%The shear velocity $u_*$ is obtained from the relationship for bed shear stress under steady state conditions:
%
%\begin{equation}
%\tau_b = \rho_w {u_*}^2
%\end{equation}
%
%\noindent where $\tau_b$ is the shear stress of the flow on the bed, which can also be written as $\tau_b = \rho g h S$, where $S$ is the slope of the bed. Combining these expressions, we can write an equation for the shear velocity:
%
%\begin{equation}
%u_* = \sqrt{g S h}
%\end{equation}

The entrainment flux $\dot{E}$ is (e.g. Hanson, 1990):

\begin{equation}
\dot{E} = K_e \: (\tau - {\tau_c})
\end{equation}

\noindent where $K_e$ is an erodibility parameter and ${\tau_c}$ is the critical shear stress. The erodibility parameter $K_e$ is given by (Hanson and Simon, 2001):
 
\begin{equation}
K_e = \frac{0.2 x 10^{-6}}{\tau_c^{0.5}}
\end{equation}

$\tau$ is the shear stress that the flow applies on the bed:

\begin{equation}
\tau = \rho_w {u_*}^2
\end{equation}

\noindent where $u_*$ is the shear velocity and $\rho_w$ is the fluid density.

The shear velocity $u_*$ is obtained by acknowledging that $\tau_b = \rho g h S$, where $S$ is the slope of the bed. We can write an equation for the shear velocity:

\begin{equation}
u_* = \sqrt{g S h}
\end{equation}


%The shear velocity $u_*$ is given by:
%
%\begin{equation}
%u_* = \frac{\langle u \rangle \kappa}{\log{(h / z_o)} - 1}
%\end{equation}
%
%\noindent where $\langle u \rangle$ is the depth-averaged flow velocity, $\kappa$ is Von Karman's constant ($\kappa = 0.408$) and $z_o = D_{50}/30$. The expression for $u_*$ is derived from the Law of the Wall equation for depth-averaged flow velocity:
%
%\begin{equation}
%\langle u \rangle = \frac{u_*}{\kappa} \bigg[ \ln{\bigg(\frac{h}{z_o}\bigg)} - 1 \bigg]
%\end{equation}

\subsection{Deposition}

Following Davy and Lague (2009), the rate of aggradation $\dot{D}$ can be written as

\begin{equation}
\dot{D} = d^* \: C \: v_s
\end{equation}

\noindent where $d^* = C^* / C$, $C^*$ is the sediment concentration at the bed interface and $C$ is the average sediment concentration in the water column. See Davy and Lague (2009) for a description of the calculation of $d^*$ in natural rivers. In practice, The value of $d^*$ will be between 1 and 3 for Rouse numbers smaller than 0.1, and it will be closer to 1 for large rivers or small particles. In small river with large particles, where much of the entrainment mechanism is bed load, $d_*$ will be much larger than 1. Our method for finding $d^*$ is described below.

The term $v_s$ is the settling velocity of grains in the fluid \citep{ferguson2004simple}:

\begin{equation}
v_s = \frac{R \: g \: {D_{50}}^2}{C_1 \, \nu + (0.75 \, C_2 \, R \, g \, {D_{50}}^3)^2}
\end{equation}

\noindent where $C_1$ and $C_2$ are constants, and $\nu$ is the kinematic viscosity of water. 

\subsection{Determining the value of $d^*$}
The selection of a value of $d^*$ for a natural river requires knowing both the velocity profile and sediment concentration profile of a river. Davy and Lague (2009) derive an expression for $d^*$ from the Rouse-Vanoni profile and the assumption that the sediment discharge of the flow $q_s$ is the integral of the concentration and flow velocity with depth:

\begin{equation}
\begin{split}
d^* & = \frac{C_s(a)}{C_s} = C_s(a) \frac{q}{q_s} \\
& = \frac{\int_{a}^{h} u(z) dz}{\int_{a}^{h} (\frac{z-a}{h-a}\frac{a}{z})^Z u(z) dz} \\
& = \frac{\int_{a}^{h} \ln{(\frac{z}{z_o})} dz}{\int_{a}^{h} (\frac{z-a}{h-a}\frac{a}{z})^Z \ln{(\frac{z}{z_o})} dz}
\end{split}
\end{equation} 

\noindent where $Z$ is the Rouse number, given by

\begin{equation}
Z = \frac{v_s}{\kappa u_*}
\end{equation}

We solve this expression using numerical integration to find the value of $d^*$ at every cell, assuming $z_o = D_{50}/30$.

\section{Vegetation}

The drag that vegetation imparts on the flow can dramatically lower the flow velocity and reduce the boundary shear stress, limiting erosion of the surface and potentially causing particles in transport to settle from suspension (Li and Shen, 1973; Pasche and Rouve, 1985; L?opez and Garc??a, 1998; Jordanova and James, 2003). A frequently used method for modeling vegetation uses a modified Manning?s equation to account for the increased roughness seen by the flow (e.g. Guardo and Tomasello, 1995). This approach does not consider the drag imparted by vegetation on the body of the flow and the quantifiable differences between various vegetation types (e.g. Kadlec, 1990). To counter these problems, other approaches treat vegetation as objects, most often cylinders, that flow must go around (e.g. Burke and Stolzenbach, 1983; Nepf , 1999).

We follow a similar approach by Kean and Smith (2004) to calculate the drag force FD that vegetation imparts on the flow:

\begin{equation}
F_D = \frac{1}{2} \rho \overline{C_D} \alpha {U_{ref}}^2
\end{equation}

\noindent where $\rho$ is the density of the fluid, $U_{ref}$ is the flow velocity in the absence of vegetation, $\alpha=d_s/\lambda^2$ is the projected plant area per unit volume, $d_s$ is the stem diameter, $\lambda$ is the mean stem spacing and $\overline{C_D}$ is the bulk drag coefficient for the vegetation array. If vegetation is approximated as a regular array of emergent cylindrical stems, $\overline{C_D} = 1.2$. A method for calculation the value of $\overline{C_D}$ as a function of $ad$ is described below.

The drag force is allowed to reduce the flow velocity to zero but not to change its direction. The velocity of the flow is reduced by the drag force using the following relationship:

\begin{equation}
U = U_{ref} - F_D \Delta t
\end{equation}

\noindent where $\Delta t$ is the model timestep.

\subsection{Drag coefficient}

Following Nepf (1999), we expand the cylinder-drag vegetation model by defining the effects of stem population density on drag coefficient. Nepf (1999) presents a force balance that is used to solve for the bulk drag coefficient $\overline{C_D}$:

\begin{equation}
(1 - a d) C_B U^2 + \frac{1}{2} \overline{C_D} a d \big(\frac{h}{d}\big) U^2 = g h \frac{\partial h}{\partial x}
\end{equation}

\noindent where $C_B$ is the bed drag coefficient [e.g. Munson et al., 1990, p. 673]. The term $\frac{\partial h}{\partial x}$ depends on $a d$. We selected the value of this term at multiple points in the range of valid values of $a d$ such that they matched the values of $\overline{C_D}$ in Figure 6 of Nepf (1999) and fit a cubic equation to this term. Solving the force balance above with this fit and quantities listed for Nepf (1999) experiments, we found a relationship between $\overline{C_D}$ and $a d$ that matches the curve in Figure 6 of Nepf (1999):

\[
 \overline{C_D} = 
  \begin{cases} 
   1.2 & \text{if } a d \leq 0.006 \\
   56.11\;(ad)^2 - 15.28\;ad + 1.3 - \num{5.465e-4}\;(ad)^{-1}       & \text{if } ad > 0.006
  \end{cases}
\]

The value of the drag coefficient $\overline{C_D}$ is calculated for every cell in the domain when the quantities for stem spacing and stem diameter are set.

\subsection{Turbulence}

By default, ANUGA assumed that fluid is inviscid and does not include kinematic viscosity. We incorporate kinematic viscosity only for vegetated systems using the formulations described here. We did not include this term for unvegetated channels with the assumption that the contribution was minimal.

"In addition to affecting the mean velocity, vegetation also affects the turbulent intensity and the diffusion. The conversion of mean kinetic energy to turbulent kinetic energy wihin stem wakes augments the turbulence intensity, and because wake turbulence is generated at the stem scale, the dominant turbulent lengthscale is shifted downward relative to unvegetated, open-channel conditions [Nepf 1997]". The formulation of Nepf (1999) for turbulent diffusivity within vegetated flows also incorporates a term for mechanical diffusivity, caused by the routing of parcels of flow along different paths between stems. We start by calculating a turbulence intensity that is given by the balance of the work input for wake production and the viscous dissipation rate. The work input for wake production is given by:

\begin{equation}
Pw = \frac{1}{2}\overline{C_D} \alpha U_{ref}^2
\end{equation}

The viscous dissipation rate is:

\begin{equation}
\varepsilon ~ k^{3/2} d_s^{-1}
\end{equation}

\noindent where $k$ is the turbulence intensity. The balance of these two equations gives:

\begin{equation}
\frac{\sqrt{k}}{U_{ref}} = a\big[\overline{C_D} ad\big]^{1/3}
\end{equation}

\noindent where $a$ is a coefficient ($a\sim1$). The turbulence intensity increases with bulk drag coefficient and with population density. Because the bulk drag coefficient $\overline{C_D}$ is a function of $ad$, the turbulence intensity can be considered to be only a function of population density.

The turbulent kinetic energy within the stems can be written as:

\begin{equation}
k = ((1 - ad) C_B + (\overline{C_D} ad)^{2/3} U^2
\end{equation}

The total diffusivity $D$ for a vegetated channel, including both turbulent and mechanical diffusion, can then be written as:

\begin{equation}
D \sim k^{1/2} \ell + [ad]Ud
\end{equation}

\noindent where $\ell$ is the mixing length scale. The mixing length scale is assumed to be equal to the flow depth in unvegetated channels ($ad = 0$). "If only sparse vegetation is present in the flow, that is, vegetation with spacing $\Delta S > h$, where $\Delta S$ is the stem spacing, the channel-scale eddies persist and continue to dminate the diffusive transport. Thus for sparse vegetation the dominant mixing length is still equal to the flow depth. For dense populations $\Delta S < h$, the stems break apart channel scale eddies, reducing the mixing-length scale, until at $ad > 0.01$, $\ell \sim d$ [Nepf et al., 1997]. This model assumes that $\ell$ varies linearly from $h$ to $d$ between these limits."

The model for mixing length looks like:

\[
 \ell = 
  \begin{cases} 
   h & \text{if } \Delta S \geq h \\
   K_\ell ad - K_\ell \big(d/h\big)^2 + h    & \text{if } \Delta S > h \\
   d & \text{if } ad \geq 0.01
  \end{cases}
\]

\noindent where $K_\ell$ is:

\begin{equation}
K_\ell = \frac{d-h}{0.01 - (d/h)^2}
\end{equation}

\section{Default values}

Some of the parameters in the equations described above were set to default values. The table below sumarizes those values and provides references:

\begin{table}[]
\centering
\caption{Table of default values}
\begin{tabular}{|l|l|l|l|l|}
\ \\ \cline{1-5}
$\rho_w$ & 1000 & kg m$^{-3}$ & Density of fluid           &  \\ \cline{1-5}
$\rho_s$ & 2650 & kg m$^{-3}$ & Density of sediment        &  \\ \cline{1-5}
$g$      & 9.81 & m s$^{-2}$  & Gravitational acceleration &  \\ \cline{1-5}
${K_e}^*$         &   2   & - & Dimensionless erosion coefficient &  ?\\ \cline{1-5}
$C_1$         &  18    &       -      & (for smooth spheres)                           & \citep{ferguson2004simple} \\ \cline{1-5}
$C_2$         &    0.4  &       -      & (for smooth spheres)                            & \citep{ferguson2004simple} \\ \cline{1-5}
$\nu$         &  $1$x${10}^{-6}$    &   m$^2$ s$^{-1}$          &   Kinematic viscosity of water                         &  \\ \cline{1-5}
$\phi$         &   0.3   &     -        & Porosity of the bed  & &\\ \cline{1-5}
${\tau_c}^*$         &   0.06   &     -        & Dimensionless critical shear stress     & e.g. Munson et al., 1990, p. 673 &\\ \cline{1-5}
$C_B$         &   0.001   &     -        & Bed drag coefficient                           & e.g. Munson et al., 1990, p. 673 &\\ \cline{1-5}
\end{tabular}
\end{table}

The default value of $D_{50} = 0.07 mm$ comes from Griffin et al., 2010.

\end{document}

















