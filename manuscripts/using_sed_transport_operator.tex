\documentclass[11pt]{article}
\begin{document}

\title{Using the Sediment Transport Operator}
\maketitle

Large floods can alter the shape of an easily erodible surface rapidly enough to drastically change the flow field during the event. Simulating these events therefore demands the dynamic evolution of the model topography in response to the computed flow and sediment transport.

The strong spatial variability in shear stress caused by complex flow patterns and variability in drag resulting from spatially variable vegetation cover requires the calculation of an explicit sediment mass balance within the water column in order to handle the local disequilibria between grain entrainment and settling rates \citep{davy2009fluvial}. The mass balance for sediment in the water column and moving bed layer is written as

\begin{equation}
\frac{\partial C h}{\partial t} = \dot{E} - \dot{D} - \left(\frac{\partial q_{s_x}}{\partial x} + \frac{\partial q_{s_y}}{\partial y} \right)
\end{equation}

\noindent where $\dot{E}$ is entrainment flux, $\dot{D}$ is deposition flux, $C$ is sediment concentration, and $q_{s_x}$ and $q_{s_y}$ are specific volumetric sediment fluxes in the $x$ and $y$ directions \citep[e.g.,][]{davy2009fluvial}. Sediment flux is given by $q_{s_{x,y}} = \beta C q_{x,y}$, where $q_{x,y}$ is specific discharge in the $x$ or $y$ direction and $\beta$ ($\le 1$) describes the effective speed of sediment relative to that of the water.

The elevation of the surface $z$ is given by

\begin{equation}
\frac{\partial z}{\partial t} = \frac{\dot{D} - \dot{E}}{1 - \phi}
\end{equation}

\noindent where $\phi$ is the porosity of the bed material.


\end{document}
