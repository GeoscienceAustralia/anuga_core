\documentclass[10pt]{article}

\usepackage[left=2cm,top=2cm,right=2cm,head=2cm,bottom=2cm,foot=1cm]{geometry}
\usepackage[square]{natbib}
\usepackage{amsmath, amsthm}
\usepackage{url}

\usepackage[utf8]{inputenc}
 
\usepackage{listings}
\usepackage{color}
\usepackage{booktabs}
\newcommand{\ra}[1]{\renewcommand{\arraystretch}{#1}}
 
\definecolor{codegreen}{rgb}{0,0.6,0}
\definecolor{codegray}{rgb}{0.5,0.5,0.5}
\definecolor{codepurple}{rgb}{0.58,0,0.82}
\definecolor{backcolour}{rgb}{0.95,0.95,0.95}
 
\lstdefinestyle{mystyle}{
    backgroundcolor=\color{backcolour},   
    commentstyle=\color{codegreen},
    keywordstyle=\color{magenta},
    numberstyle=\tiny\color{codegray},
    stringstyle=\color{codepurple},
    basicstyle=\ttfamily\footnotesize,
    breakatwhitespace=false,         
    breaklines=true,                 
    captionpos=b,                    
    keepspaces=true,                 
    %numbers=left,                    
    numbersep=5pt,                  
    showspaces=false,                
    showstringspaces=false,
    showtabs=false,                  
    tabsize=4,
    showlines=true
}
 
 
\lstset{style=mystyle}


\begin{document}


\title{Using the Sediment Transport and Vegetation Operators in ANUGA}
\author{Mariela Perignon}
\date{\today}
\maketitle

This document describes the implementation and use of operators for sediment transport and vegetation drag. These operators are still preliminary and considered a test release.

The sediment transport operator can currently only be used with a single processor. Making it parallel safe would require modifying, at minimum, the algorithm for calculating sediment flux. The vegetation drag operator should be parallel safe but has not been tested with multiple processors.

The mathematical framework behind these operattors is described in \citet{Simpson:2006aa} and \citet{davy2009fluvial} for calculating sediment transport and momentum sinks, and \citet{nepf1999drag} and \citet{kean2006form} for vegetation drag.

\section{Installation}

A script for installing these operators and the associated files is included in the download. Run the scripts to place the operator files in your existing installation of anuga.

\section{Files}

The two operators are located in the directory \url{operators}, this documentation is in the directiory \url{doc}, and example files for using these two operators are in the directory \url{examples/operators/sed}.

\section{Sediment Transport Operator}

\subsection{Creating the domain}\label{domain}

In order to use the sediment transport operator, the quantity \textit{concentration} must be defined as an evolved quantity when the domain is created. Exceptions are raised if this quantity does not exist when the operator is first called during the run.
\ \\

The existing function for creating a rectangular domain accepts a list of evolved quantities as an argument:

% Creating a rectangular domain %
\begin{minipage}[c]{0.95\textwidth}
\begin{lstlisting}[language=Python, title=Creating a rectangular domain]

evolved_quantities = ['stage', 'xmomentum', 'ymomentum', 'concentration']
			  
points, vertices, boundary = rectangular_cross(int(length/dx),
									    	   int(width/dy),
									    	   len1=length,
									    	   len2=width)
                                                  	
domain = Domain(points, vertices, boundary,
				evolved_quantities = evolved_quantities)

\end{lstlisting}
\end{minipage}
\ \\

Creating a domain that uses a mesh as input for \textit{elevation} is usually done with the function \url{create_domain_from_regions}, which does not accept a list of evolved quantities. To accomplish the same functionality, use \url{create_mesh_from_regions} to generate the mesh and then create the domain:

% Creating a domain from regions %
\begin{minipage}[c]{0.95\textwidth}
\begin{lstlisting}[language=Python, title=Creating a domain from a mesh file]

from anuga.pmesh.mesh_interface import create_mesh_from_regions

create_mesh_from_regions(bounding_polygon = bounding_polygon,
                         boundary_tags = boundary_tags,
                         maximum_triangle_area = 200,
                         filename = 'topo.msh')

evolved_quantities =  ['stage', 'xmomentum', 'ymomentum', 'concentration']

domain = Domain(filename_root + '.msh', evolved_quantities = evolved_quantities)


\end{lstlisting}
\end{minipage}
\ \\ \ \\

Because it updates the elevation of the bed, the sediment transport operator can only be used with one of ANUGA's discontinuos elevation ('DE') flow algorithms. Changes in the elevation of the bed and the distribution of sediment in transport during the simulation can be recorded in the SWW file by including \url{elevation} and \url{concentration} in \url{domain.set_quantities_to_be_stored} with a value of 2 (to save them at every yieldstep):

% Recording output %
\begin{minipage}[c]{0.95\textwidth}
\begin{lstlisting}[language=Python, title=Setting the flow algorithm and recording the output]

domain.set_flow_algorithm('DE0')

domain.set_quantities_to_be_stored({'elevation': 2,
                                    'stage': 2,
                                    'xmomentum': 2,
                                    'ymomentum': 2,
                                    'concentration': 2})

\end{lstlisting}
\end{minipage}

\subsection{Initializing the operator}

The sediment transport operator is initialized in the run file with:

% Creating a sed transport operator %
\begin{minipage}[c]{0.95\textwidth}
\begin{lstlisting}[language=Python, title=Initializing the sediment transport operator]
 
from anuga.operators.sed_transport_operator import Sed_transport_operator

sed_op = Sed_transport_operator(domain)

\end{lstlisting}
\end{minipage}
\ \\

\begin{table}[h]
\centering
\ra{1.3}
\begin{tabular}{@{}rcll@{}}
\toprule
Variable name & Default value & Units & Quantity  \\ \midrule
\verb!rho_w! & 1000 & kg m$^{-3}$ & Density of fluid  \\
\verb!rho_s! & 2650 & kg m$^{-3}$ & Density of sediment \\
\verb!nu!         &  $1$x${10}^{-6}$    &   m$^2$ s$^{-1}$          &   Kinematic viscosity of water       \\
\verb!porosity!         &   0.3   &     -        & Porosity of the bed  \\
\verb!criticalshear_star!         &   0.06   &     -        & Dimensionless critical shear stress    \\
\verb!grain_size!         &   0.00013   &     m        & Grain size       \\
\bottomrule
\end{tabular}
\caption{Default values for equation parameters in Sediment transport operator}
\label{table:sed_transport_parameters}
\end{table}

Table \ref{table:sed_transport_parameters} summarizes the default values that various equation parameters take when \url{Sed_transport_operator} is initialized. These values can be modified from the run file after the operator is initialized.
\ \\

The initial concentration of sediment in the flow within the domain can be set at any point after the domain is created with \url{domain.set_quantity()} (as a fraction of the volume of the water column). The concentration of flow entering the domain through any Dirichlet boundary can be set after initializing the operator as the operator variable \url{inflow_concentration}. If not set, \url{inflow_concentration} takes the maximum value of the quantity \url{concentration} at the first timestep. Currently, the value of \url{inflow_concentration} applies to all Dirichlet boundaries in the domain.

% Creating a sed transport operator %
\begin{minipage}[c]{0.95\textwidth}
\begin{lstlisting}[language=Python, title=Modifying parameter values]
 
domain.set_quantity('concentration', 0.01) # 1 percent

sed_op.inflow_concentration = 0.02 # 2 percent
 
sed_op.grain_size = 0.0002 # 0.2 mm grains
sed_op.criticalshear_star = 0.03

\end{lstlisting}
\end{minipage}
\ \\

\subsection{Mathematical Background and Assumptions}

The sediment transport operator acts only on cells with a flow depth greater than a minimum depth (5 cm) and non-zero x-directed momentum. This avoids extremely high erosion rates due to abnormally high velocities at low flow depths and minimizes the likelihood of aggradation that exceed the water depth.
\ \\

The strong spatial variability in shear stress caused by complex flow patterns requires the calculation of an explicit sediment mass balance within the water column in order to handle the local disequilibria between grain entrainment and settling rates \citep{davy2009fluvial}. The mass balance for sediment in the water column and moving bed layer is written as

\begin{equation}
\frac{\partial C h}{\partial t} = \dot{E} - \dot{D} - \left(\frac{\partial q_{s_x}}{\partial x} + \frac{\partial q_{s_y}}{\partial y} \right)
\end{equation}

\noindent where $\dot{E}$ is the entrainment flux, $\dot{D}$ is the deposition flux, $C$ is sediment concentration (as a fraction of the volume of the water column), and $q_{s_x}$ and $q_{s_y}$ are specific volumetric sediment fluxes in the $x$ and $y$ directions \citep[e.g.,][]{davy2009fluvial}. Sediment flux is given by $q_{s_{x,y}} = \beta C q_{x,y}$, where $q_{x,y}$ is specific discharge in the $x$ or $y$ direction and $\beta$ ($\le 1$) describes the effective speed of sediment relative to that of the water.

The rate of change of elevation of the bed is a mass balance between material entrained and deposited:

\begin{equation}
\frac{\partial z}{\partial t} = \frac{\dot{D} - \dot{E}}{1 - \phi}
\end{equation}

\noindent where $z$ is the elevation of the bed above some datum, and $\phi$ is the porosity of the bed material.

\subsubsection{Entrainment}:

The entrainment flux $\dot{E}$ is given by \citep[e.g.,][]{hanson1990surface}:

\begin{equation}
\dot{E} = K_e \: (\tau - {\tau_c})
\end{equation}

\noindent where $K_e$ is an erodibility parameter and ${\tau_c}$ is the critical shear stress for particle entrainment. The erodibility parameter $K_e$ can be written as \citep{hanson2001erodibility}:
 
\begin{equation}
K_e = \frac{0.2 x 10^{-6}}{\tau_c^{0.5}}
\end{equation}

The critical shear stress ${\tau_c}$ can be found from the dimensionless critical shear stress ${\tau^*_c}$ through:

\begin{equation}
\tau^*_c = \frac{\tau_c}{(\rho_s - \rho_w) g D_{50}}
\end{equation}

\noindent where $\rho_w$ and $\rho_s$ are the density of the fluid and sediment, $g$ is acceleration due to gravity and $D_{50}$ is the median grain size of the bed material.
 
$\tau_b$ is the shear stress that the flow applies on the bed:

\begin{equation}
\tau_b = \rho_w {u_*}^2
\end{equation}

\noindent where $u_*$ is the shear velocity. Shear stress $\tau$ can also be written as $\tau_b = \rho g h S$, where $S$ is the local slope of the bed. An equation for shear velocity $u_*$ can then be obtained:

\begin{equation}
u_* = \sqrt{g S h}
\end{equation}

\subsubsection{Aggradation}:

Following \citet{davy2009fluvial}, the rate of aggradation $\dot{D}$ can be written as:

\begin{equation}
\dot{D} = d^* \: C \: v_s
\end{equation}

\noindent where $d^* = C^* / C$ is a dimensionless number that describes the vertical distribution of sediment in the water column, $C^*$ is the sediment concentration at the bed bed interface and $C$ is the average sediment concentration in the water column. The value of $d^*$ in natural rivers will be between 1 and 3 for Rouse numbers smaller than 0.1, and close to 1 for large rivers or small particles. In small river with large particles, where much of the entrainment mechanism is bed load, $d_*$ will be much larger than 1. Our method for finding $d^*$ is described below.

The term $v_s$ is the settling velocity of grains in the fluid \citep{ferguson2004simple}:

\begin{equation}
v_s = \frac{R \: g \: {D_{50}}^2}{C_1 \, \nu + (0.75 \, C_2 \, R \, g \, {D_{50}}^3)^2}
\end{equation}

\noindent where $C_1$ and $C_2$ are constants related to the shape and roughness of the particles and $\nu$ is the kinematic viscosity of water. 

\paragraph{Selecting a value for $d^*$}

Calculating $d^*$ for a natural river requires knowing both the velocity profile and sediment concentration profile in the flow. \citet{davy2009fluvial} derive an expression for $d^*$ from the Rouse-Vanoni profile and the assumption that the sediment discharge of the flow $q_s$ is the integral of the concentration and flow velocity with depth:

\begin{equation}
\begin{split}
d^* & = \frac{C_s(a)}{C_s} = C_s(a) \frac{q}{q_s} \\
& = \frac{\int_{a}^{h} u(z) dz}{\int_{a}^{h} (\frac{z-a}{h-a}\frac{a}{z})^Z u(z) dz} \\
& = \frac{\int_{a}^{h} \ln{(\frac{z}{z_o})} dz}{\int_{a}^{h} (\frac{z-a}{h-a}\frac{a}{z})^Z \ln{(\frac{z}{z_o})} dz}
\end{split}
\end{equation} 

\noindent where $Z$ is the Rouse number, given by

\begin{equation}
Z = \frac{v_s}{\kappa u_*}
\end{equation}

We solve this expression using numerical integration to find the value of $d^*$ at every cell, assuming $z_o = D_{50}/30$ and a flow depth of 1 meter. For computational speed, values of $d^*$ for all cells in the domain are only calculated once every 10 timesteps.

\section{Vegetation}

\subsection{Initializing the operator}

The vegetation operator is initialized in the run file with:

% Creating a veg operator %
\begin{minipage}[c]{0.95\textwidth}
\begin{lstlisting}[language=Python, title=Initializing the vegetation operator]
 
from anuga.operators.vegetation_operator import Vegetation_operator

veg_op = Vegetation_operator(domain)

\end{lstlisting}
\end{minipage}
\ \\

The vegetation operator uses quantities \url{veg_diameter} and \url{veg_spacing} for calculating the drag that vegetation applies on the flow. These quantities can be created at any point before the run starts by writing:

% Creating veg quantities %
\begin{minipage}[c]{0.95\textwidth}
\begin{lstlisting}[language=Python, title=Creating vegetation quantities]

Quantity(domain, name='veg_diameter', register=True)
domain.set_quantity('veg_diameter', 0.00064) # meters

Quantity(domain, name='veg_spacing', register=True)
domain.set_quantity('veg_spacing', 0.15) # meters

\end{lstlisting}
\end{minipage}
\ \\

The argument \url{register=True} is required for the quantities to be available to the operators.

\subsection{Mathematical Background and Assumptions}

The drag that vegetation imparts on the flow can dramatically lower the flow velocity and reduce the boundary shear stress, limiting erosion of the surface and potentially causing particles in transport to settle from suspension (Li and Shen, 1973; Pasche and Rouve, 1985; L?opez and Garc??a, 1998; Jordanova and James, 2003). A frequently used method for modeling vegetation uses a modified Manning?s equation to account for the increased roughness seen by the flow (e.g. Guardo and Tomasello, 1995). This approach does not consider the drag imparted by vegetation on the body of the flow and the quantifiable differences between various vegetation types (e.g. Kadlec, 1990). To counter these problems, other approaches treat vegetation as objects, most often cylinders, that flow must go around (e.g. Burke and Stolzenbach, 1983; Nepf , 1999).

We follow a similar approach by Kean and Smith (2004) to calculate the drag force FD that vegetation imparts on the flow:

\begin{equation}
F_D = \frac{1}{2} \rho \overline{C_D} \alpha {U_{ref}}^2
\end{equation}

\noindent where $\rho$ is the density of the fluid, $U_{ref}$ is the flow velocity in the absence of vegetation, $\alpha=d_s/\lambda^2$ is the projected plant area per unit volume, $d_s$ is the stem diameter, $\lambda$ is the mean stem spacing and $\overline{C_D}$ is the bulk drag coefficient for the vegetation array. If vegetation is approximated as a regular array of emergent cylindrical stems, $\overline{C_D} = 1.2$. A method for calculation the value of $\overline{C_D}$ as a function of $ad$ is described below.

The drag force is allowed to reduce the flow velocity to zero but not to change its direction. The velocity of the flow is reduced by the drag force using the following relationship:

\begin{equation}
U = U_{ref} - F_D \Delta t
\end{equation}

\noindent where $\Delta t$ is the model timestep.

\subsection{Drag coefficient}

Following Nepf (1999), we expand the cylinder-drag vegetation model by defining the effects of stem population density on drag coefficient. Nepf (1999) presents a force balance that is used to solve for the bulk drag coefficient $\overline{C_D}$:

\begin{equation}
(1 - a d) C_B U^2 + \frac{1}{2} \overline{C_D} a d \big(\frac{h}{d}\big) U^2 = g h \frac{\partial h}{\partial x}
\end{equation}

\noindent where $C_B$ is the bed drag coefficient [e.g. Munson et al., 1990, p. 673]. The term $\frac{\partial h}{\partial x}$ depends on $a d$. We selected the value of this term at multiple points in the range of valid values of $a d$ such that they matched the values of $\overline{C_D}$ in Figure 6 of Nepf (1999) and fit a cubic equation to this term. Solving the force balance above with this fit and quantities listed for Nepf (1999) experiments, we found a relationship between $\overline{C_D}$ and $a d$ that matches the curve in Figure 6 of Nepf (1999):

\[
 \overline{C_D} = 
  \begin{cases} 
   1.2 & \text{if } a d \leq 0.006 \\
   56.11\;(ad)^2 - 15.28\;ad + 1.3 - \num{5.465e-4}\;(ad)^{-1}       & \text{if } ad > 0.006
  \end{cases}
\]

The value of the drag coefficient $\overline{C_D}$ is calculated for every cell in the domain when the quantities for stem spacing and stem diameter are set.

\subsection{Turbulence}

By default, ANUGA assumed that fluid is inviscid and does not include kinematic viscosity. We incorporate kinematic viscosity only for vegetated systems using the formulations described here. We did not include this term for unvegetated channels with the assumption that the contribution was minimal.

"In addition to affecting the mean velocity, vegetation also affects the turbulent intensity and the diffusion. The conversion of mean kinetic energy to turbulent kinetic energy wihin stem wakes augments the turbulence intensity, and because wake turbulence is generated at the stem scale, the dominant turbulent lengthscale is shifted downward relative to unvegetated, open-channel conditions [Nepf 1997]". The formulation of Nepf (1999) for turbulent diffusivity within vegetated flows also incorporates a term for mechanical diffusivity, caused by the routing of parcels of flow along different paths between stems. We start by calculating a turbulence intensity that is given by the balance of the work input for wake production and the viscous dissipation rate. The work input for wake production is given by:

\begin{equation}
Pw = \frac{1}{2}\overline{C_D} \alpha U_{ref}^2
\end{equation}

The viscous dissipation rate is:

\begin{equation}
\varepsilon ~ k^{3/2} d_s^{-1}
\end{equation}

\noindent where $k$ is the turbulence intensity. The balance of these two equations gives:

\begin{equation}
\frac{\sqrt{k}}{U_{ref}} = a\big[\overline{C_D} ad\big]^{1/3}
\end{equation}

\noindent where $a$ is a coefficient ($a\sim1$). The turbulence intensity increases with bulk drag coefficient and with population density. Because the bulk drag coefficient $\overline{C_D}$ is a function of $ad$, the turbulence intensity can be considered to be only a function of population density.

The turbulent kinetic energy within the stems can be written as:

\begin{equation}
k = ((1 - ad) C_B + (\overline{C_D} ad)^{2/3} U^2
\end{equation}

The total diffusivity $D$ for a vegetated channel, including both turbulent and mechanical diffusion, can then be written as:

\begin{equation}
D \sim k^{1/2} \ell + [ad]Ud
\end{equation}

\noindent where $\ell$ is the mixing length scale. The mixing length scale is assumed to be equal to the flow depth in unvegetated channels ($ad = 0$). "If only sparse vegetation is present in the flow, that is, vegetation with spacing $\Delta S > h$, where $\Delta S$ is the stem spacing, the channel-scale eddies persist and continue to dminate the diffusive transport. Thus for sparse vegetation the dominant mixing length is still equal to the flow depth. For dense populations $\Delta S < h$, the stems break apart channel scale eddies, reducing the mixing-length scale, until at $ad > 0.01$, $\ell \sim d$ [Nepf et al., 1997]. This model assumes that $\ell$ varies linearly from $h$ to $d$ between these limits."

The model for mixing length looks like:

\[
 \ell = 
  \begin{cases} 
   h & \text{if } \Delta S \geq h \\
   K_\ell ad - K_\ell \big(d/h\big)^2 + h    & \text{if } \Delta S > h \\
   d & \text{if } ad \geq 0.01
  \end{cases}
\]

\noindent where $K_\ell$ is:

\begin{equation}
K_\ell = \frac{d-h}{0.01 - (d/h)^2}
\end{equation}


\bibliography{/Users/mari/Dropbox/Papers/library}
\bibliographystyle{agu08}

\end{document}

%%%%%%%%%%%%%%%%%%%%%%%%%%%%%%%%%%%%%%%%%%%%%%%%%%%%%%%%%%%%%%%%%%%%%%%%%%%%%%%%%%%%%%%%%%
%%%%%%%%%%%%%%%%%%%%%%%%%%%%%%%%%%%%%%%%%%%%%%%%%%%%%%%%%%%%%%%%%%%%%%%%%%%%%%%%%%%%%%%%%%
\subsubsection{Sources and sinks of momentum}

Four sources or sinks of momentum can be included by these operators into the equations for conservation of momentum: (1) friction loss, (2) loss due to spatial variations in sediment concentration, (3) loss of momentum due to the exchange of mass between the flow and the bed, and (4) loss due to fluid drag on vegetation. Changes to the momentum of the flow due to turbulence are calculated when the function argument \textit{turbulence} is True.

\begin{enumerate}
\item Loss due to friction is computed using the existing \url{manning_friction_implicit} function

\item Loss due to spatial variations in concentration is given by the equation:

\begin{equation}
P_C = \frac{(\rho_s - \rho_w) g h^2}{2 (\rho_w (1 - C) + \rho_s C)} \nabla \cdot C
\end{equation} 

\item Loss due to the exchange of mass between the flow and the bed is given by:

\begin{equation}
P_e = \frac{\dot{D} - \dot{E}}{1 - \phi} \left(\frac{\rho_w \phi + \rho_s (1 - \phi)}{\rho_w (1 - C) + \rho_s C} -1 \right) \vec{v}
\end{equation}

\noindent where $\vec{v}$ is the flow velocity.

\item Loss of momentum due to fluid drag on vegetation is given by:

\begin{equation}
P_d = - \vec{v} \vec{F_D} h
\end{equation}

\noindent where $\vec{F_D}$ is the drag force of vegetation on the flow. The quantities \textit{xmomentum} and \textit{ymomentum} are modified by this equation whenever the \url{Vegetation_operator} is used, regardless of the \url{mometum_sinks} argument.

\end{enumerate}

\subsubsection{Updating quantities}

All operators listed in \url{fractional_step_operators} are called at every timestep by \url{domain.evolve} after the flow calculations have been performed and the conserved quantities updated.
\ \\


Both the \url{Sed_transport_operator} and \url{Vegetation_operator} directly modify the values of several quantities through functions in \url{sed_transport_mesh.py}.

\begin{itemize}
\item \textit{Momentum} is modified by using the \url{explicit_update} (re-set to zero) and \url{update} methods of the quantities \textit{xmomentum} and \textit{ymomentum}.

\item The shape of the bed is altered by using the \url{set_values} function at the vertices for the quantity \textit{elevation}, followed by \url{smooth_vertex_values()} to eliminate discontinuities in the topography and distribute the updated elevations to the centroids and edges.

\item The flux of sediment across the cell edges is computed by (1) obtaining the total volume of sediment in the flow within each cell from the values of the quantity \textit{concentration} at the centroids, (2) finding the normal component of the flow velocities at the edges of the cells, (3) calculating, from the values of \textit{concentration} at the edges, the volume of sediment that crosses each edge within the timestep, and (4) integrating the flux in and out of each cell across all edges to (5) change the volume of sediment present in the flow within each cell. This volume is then (6) converted back to a sediment concentration (of the form $C h$) at the centroid of each cell, and (7) the updated value at each vertex is calculated by averaging the value at the centroids around it. (8) The quantity \textit{concentration} is updated using the function \url{set_values} at the vertices.
\end{itemize}

\section{Utilities} \label{utilities}

The file \url{sed_transport_utils.py} contains a set of functions, similar to existing ones, that are tailored to the needs of these specific operators.

\begin{description}
\item[\url{create_domain_from_regions_sed(...)}] \hfill \\
This function is equivalent to \url{__init__.create_domain_from_regions} but accepts lists of quantities to be created within the domain. See example in section \ref{domain}.

\item[\url{Reflective_boundary_Sed(domain)}] \hfill \\
This boundary type is equivalent to \url{Reflective_bondary} but manages the boundary value for \textit{concentration}.

\item[\url{Dirichlet_boundary_Sed([stage xmom ymom C])}] \hfill \\
This boundary is equivalent to the existing \url{Dirichlet_boundary} but accepts a fourth entry for fractional concentration ($C$, not $C h$) at the boundary. This boundary type should only be used for inflow boundaries. Flow outlets should be declared using the regular Dirichlet boundary type, which will allow sediment to be carried across the boundary by the flow instead of fixing the concentration at the cell edges. An exception is raised if a \url{Dirichlet_boundary_Sed} is has a fixed stage that is lower than the lowest elevation in the domain.
\end{description}

\begin{lstlisting}[language=Python, caption=Using operator-specific boundary types]
 
from anuga.operators.sed_transport.sed_transport_utils \
	import Reflective_boundary_Sed, Dirichlet_boundary_Sed
	
Bd = Dirichlet_boundary_Sed([1527.3, 0., 0., 0.3])	# Inlet (30% sed)
Bi = anuga.Dirichlet_boundary([1520, 0., 0.])			# Open outlet
Br = Reflective_boundary_Sed(domain)				# Reflective wall

domain.set_boundary({'bottom' : Bi,
					'side1' : Br,
					'side2' : Br,
					'top' : Bd,
					'side3' : Br,
					'side4' : Br,
					'exterior' : Br})
\end{lstlisting}

\begin{description}
\item[\url{set_quantity_NNeigh(quantity_name, filename)}] \hfill \\
Equivalent to the function \url{set_quantity}, it assigns values from the point file \textit{filename} to quantity \textit{quantity name}. In contrast to \url{set_quantity}, which interpolates between the values in the point file to obtain the value at the centroids, this function uses a Nearest Neighbour algorithm to assign each centroid the value of the nearest point in the file. This is necessary when assigning values from an ASCII file to the quantity \textit{vegetation}.
\end{description}

\noindent Two files in the directory \url{operators/sed_transport/file_conversion} can be used to transform ASCII files into point files that represent quantities other than \textit{elevation}.

\begin{description}
\item[\url{generic_asc2dem(name_in, quantity_name, ...)}] \hfill \\
Equivalent to \url{anuga.asc2dem}, this function transforms an \url{.asc} file into a {.dem} file. While the original function assumes that the values in the files will be assigned to the quantity \textit{elevation} (and codes this into the files), this function allows the user to define the name of that quantity in the argument \textit{quantity name}.

\item[\url{generic_dem2pts(name_in, quantity_name, ...)}] \hfill \\
Equivalent to \url{anuga.dem2pts}, this function transforms a \url{.dem} file into a {.pts} file for future use with the quantity \textit{quantity name}. This function does not create that quantity or set its values.
\end{description}

\begin{lstlisting}[language=Python, caption=Using file conversion and set quantity utilities]
 
from anuga.operators.sed_transport.file_conversion.generic_asc2dem
	import generic_asc2dem 
from anuga.operators.sed_transport.file_conversion.generic_dem2pts
	import generic_dem2pts
from anuga.operators.sed_transport.sed_transport_utils
	import set_quantity_NNeigh

generic_asc2dem('veg.asc',
				quantity_name = 'vegetation',
				use_cache = False,
				verbose = True)
				
generic_dem2pts('veg.dem',
				quantity_name = 'vegetation',
				use_cache = False,
				verbose = True)

set_quantity_NNeigh(domain,
					'vegetation', 
					filename='veg.pts')
    

\end{lstlisting}

\section{Tests}

A suite of unit tests and analytical solutions will be added to the operator package in the immediate future.


\subsubsection{Process-specific quantities}

\begin{description}
\item[\textit{concentration}] \hfill

Following \citet{Simpson:2006aa}, \textit{concentration} is given by:

\begin{equation}
\mbox{Concentration } = \frac{Q_s}{Q} h = C h
\end{equation}

\noindent where $Q_s$ is the sediment discharge, $Q$ is the total discharge, $h$ is the flow depth, and $C$ is the fractional concentration of sediment (by volume) in the flow. The operators store $C h$ as the quantity \textit{concentration}. It is important to note that the sediment transport-specific Dirichlet boundary takes $C$, not $C h$, as the input value.

\item[\textit{vegetation}] \hfill

The values that are stored as the quantity \textit{vegetation} are numeric codes (positive integers) for different types of vegetation, each with corresponding values of stem diameter and stem spacing in a look-up table. The default value for \textit{vegetation} is zero, which is the code for no vegetation. The values of this quantity can be set using the existing functions in ANUGA or ones included in the directory \url{/file_conversion} and the file \url{sed_transport_utils.py} (see description in section \ref{utilities}).

If any entry of \textit{vegetation} is greater than zero, the operators will attempt to import the csv file specified in the function argument \textit{vegfile} of \url{Vegetation_operator}.

An exception is raised if the file is not found or if \textit{vegfile} is not specified. In the case of a domain with two different types of vegetation ($\mbox{\textit{vegetation}}>1$) as well as bare areas ($\mbox{\textit{vegetation}}=0$), this file would contain a column \textit{vegcode} with the corresponding code in the quantity \textit{vegetation}, and two columns for the values of \textit{stem diameter} and \textit{stem spacing} (in meters).

\begin{lstlisting}[language=TeX, caption=Example of vegfile]
vegcode, stem_diameter, stem_spacing
1, 0.01, 0.1
2, 0.01, 0.3
\end{lstlisting} 

\item[\textit{diffusivity}] \hfill

The effects of turbulence on the momentum of the flow can be calculated using these operators by harnessing the existing kinematic viscosity operator (\url{domain.set_use_kinematic_viscosity}) and dynamically modifying the values of the quantity \textit{diffusivity} to reflect the conditions of the flow. An exception is raised by the operators if the argument \textit{turbulence} is True but the quantity \textit{diffusivity} does not exist. A message is then displayed and the run continues without calculating turbulence.

The values for \textit{diffusivity} are calculated using the equation:

\begin{equation}
K_d = \sqrt{k_e} \: l + ad \: U_{ref} \: d_s
\end{equation}

\noindent where $l$ is the mixing length, $U_{ref}$ is the flow velocity felt by the stems, $d_s$ is the stem diameter, $ad = d_s^2 / \Delta S^2$ is the fractional volume of the flow domain occupied by plants, and $\Delta S$ is the mean stem spacing. The velocity scale $\sqrt{k_e}$ is given by:

\begin{equation}
k_e = C_b \: U_{ref} + (\overline{C_D} \: ad)^{2/3} \: {U_{ref}}^2
\end{equation}

\noindent where $C_b$ is the bed drag coefficient ($C_b = 0.001$) and $\overline{C_D}$ is the bulk drag coefficient for the vegetation array. If the stems are approximated as emergent cylinders, $\overline{C_D} = 1.2$. The mixing length $l$ is given by:

\begin{equation}
l = h \: -  \: \frac{h - h(ad - 0.005)}{0.005}
\end{equation}

\noindent where $h$ is the flow depth. The second term of all three equations is equal to zero when vegetation is not present ($ad = 0$). This equation is an approximation of the data shown in \citet{nepf1999drag}, figure 6.

\end{description}


\bibliography{/Users/mari/Dropbox/Papers/library}
\bibliographystyle{agu08}

\end{document}
