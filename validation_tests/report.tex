\documentclass[11pt,a4paper]{report}



\usepackage{graphicx}
\usepackage{epstopdf}
\usepackage[section]{placeins} % 'one-shot' command to nicely place figures
\usepackage{datatool}


\newcommand{\anuga}{\textsc{anuga}}


\newcommand{\cwd}{}
\newcommand{\setcwd}[1]{\renewcommand{\cwd}[1]{#1##1}}


\newcommand{\inputresults}[1]{\setcwd{#1/}\graphicspath{{\cwd}}
\section{Dam Break}

Standard dam break test problem. Should show rarefaction fan and shock. 

\subsection{Results}


Here are some results.

\begin{figure}
\includegraphics{stage_plot.png}
\caption{Stage results}
\end{figure}


\begin{figure}
\includegraphics{xvel_plot.png}
\caption{Velocity results}
\end{figure}


\endinput}

%=========================================
\begin{document} 
%=========================================
\title{Automated Report on the Performance of \anuga ~on Various Test Problems}
\maketitle
\tableofcontents
%======================
\chapter{Introduction}
%======================

Here we collect the results of running our validation tests. 



%======================
\chapter{Analytical Tests}
%======================

\inputresults{Tests/Analytical/Dam_Break}

\inputresults{Tests/Analytical/runup1}

\inputresults{Tests/Analytical/runup_sinusoid}

\inputresults{Tests/Analytical/trapezoidal_channel}

\inputresults{Tests/Analytical/parabolic_basin_1D}

\inputresults{Tests/Analytical/deep_wave}

%======================
\chapter{Experimental Tests}
%======================

\inputresults{Tests/Experimental/Isolated_Building}

%======================
\chapter{Real World Tests}
%======================

\inputresults{Tests/Real_World/Patong}

%======================
\appendix
%======================
%======================
\chapter{Adding New Tests}
%======================


To setup a new validation test, create a test directory under the
\textsc{Tests} directory. In that directory there should be the test code, a
\TeX{} file \texttt{results.tex} and a python script
\texttt{produce\_results.py}, which runs the simulation and produces the
outputs. In this \TeX{} file, \texttt{report.tex}, add a line
\begin{verbatim}
\inputresults{Tests/Directory/Name}
\end{verbatim}
\end{document}
