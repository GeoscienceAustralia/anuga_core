\section{Simple wave runup}
This scenario simulates a wave flowing up a planar beach. Following the initial wave runup, eventually the water elevation should become constant, and the velocities should approach zero. 

\subsection{Results}
Figure~\ref{fig:stage_1s} shows the water surface at time 1s (in the cross-shore direction). It is not constant as the water runup at this time. Figure~\ref{fig:xvel_1s} shows the corresponding $x$-velocity during the wave runup. The velocities should be free from major spikes.
\begin{figure}[h]
\begin{center}
\includegraphics[width=0.9\textwidth]{stage_1s.png}
\caption{Water surface during the wave runup at time $t=1.0$\,.}
\label{fig:stage_1s}
\end{center}
\end{figure}

\begin{figure}[h]
\begin{center}
\includegraphics[width=0.9\textwidth]{xvel_1s.png}
\caption{Xvelocity during the wave runup at time $t=1.0$\,.}
\label{fig:xvel_1s}
\end{center}
\end{figure}



Figure~\ref{fig:stage_30s} shows the water surface at time 30s (in the cross-shore direction). It should be nearly constant (= -0.1m) in the wet portions of the domain. Figure~\ref{fig:xvel_30s} shows the corresponding velocity at time 30s. It should be nearly zero (e.g. $<<$ 1 mm/s). This case has been used to illustrate wet-dry artefacts in some versions of \anuga.
\begin{figure}[h]
\begin{center}
\includegraphics[width=0.9\textwidth]{stage_30s.png}
\caption{Water surface at time 30s after the wave runup. It should be nearly constant in wet parts of the domain.}
\label{fig:stage_30s}
\end{center}
\end{figure}

\begin{figure}[h]
\begin{center}
\includegraphics[width=0.9\textwidth]{xvel_30s.png}
\caption{Xvelocity at time 30s after the wave runup. It should be nearly zero.}
\label{fig:xvel_30s}
\end{center}
\end{figure}

\endinput